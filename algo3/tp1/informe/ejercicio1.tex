\begin{section}{Problema 1}

	\textit{Dados $b,n \in \mathbb{N} $ calcular $b^n\; mod\; n$}

	\begin{subsection}{Explicación}

		Una solución factible al problema es utilizar un algoritmo recursivo basado en la técnica $Divide \& Conquer$. La idea sería quedarnos cada vez con un problema más chico, dividiendo el exponente a la mitad y resolver recursivamente dicho problema hasta llegar a uno suficientemente chico (en este caso cuando el exponente sea 2) para luego combinar las soluciones elevando al cuadrado lo ya calculado y obtener así una solución al problema original. En el caso donde $n$ fuese impar se resuelve el problema para $n-1$ a través del procedimiento mencionado y luego se multiplica el resultado por $b$.

		Para la resolución del problema decidimos utilizar un algoritmo iterativo frente a uno recursivo debido al uso que este último hace de la pila, lo cual implica repetidos accesos a memoria que disminuyen la performance del algoritmo, y la posibilidad de {\em stack overflow}.

		La versión iterativa de la solución funciona de la siguiente manera: sea $res$ una variable de tipo número natural inicializada en 1. En cada iteración el algoritmo divide el exponente a la mitad y eleva al cuadrado $b$. Cuando llega a un exponente impar, multiplica el resultado parcial ($res$) por $b$. Cuando el exponente es 0, termina el algoritmo, siendo el valor de $res$ el resultado final. Es decir, el resultado es la productoria de los resultados parciales obtenidos en los exponentes impares. Al acumular en $b$ las potencias calculadas el algoritmo evita repetir cálculos y logra disminuir la cantidad de multiplicaciones.

		Tanto la versión recursiva como la iterativa toman módulo $n$ luego de cada multiplicación para asegurarse que el resultado entra en el tamaño de la variable (suponiendo que $(n-1)\times (n-1)$ entra).
	\end{subsection}
	
	\begin{subsection}{Detalles de la implementación}
%TODO:
%	· entra if o no entra al if
%	· caso base
%	· módulo al principio

		\begin{itemize}
			\item La función recibe como parámetros por copia $n$ y $b$.
			\item Cuando $b$ es 1, entra en un caso base y el resultado es $1\;mod\;n$ ya que 
			$1^n\;mod\;n = 1\;\forall\;n>1$ y $1\;mod\;1 = 0$.
			\item Luego de cada multiplicación toma el módulo ya que
		 	$$b^k * b^{n-k}\; mod\;n = ((b^k\;mod\;n)*(b^{n-k}\;mod\;n))\;mod\;n\;\forall\;k\leq n ^{[2]}$$ 

			De esta manera, incluso si el cálculo de $b^n$ es un número tan grande que no entra en el 
			tamaño de la variable, se va a poder realizar sin problemas (suponiendo que $(n-1)^2$ entra 
			en una variable).

		\end{itemize}
		
		\vspace{0.5cm}
		{\footnotesize [2] Por propiedades del módulo $x*y\; mod \; z = ((x\; mod\; z)*(y\; mod\; z))\;mod\; z$ } \\
		
	\end{subsection}



	\begin{subsection}{Análisis de complejidad}
		Esta vez, elegimos un modelo logaritmico para analizar el algoritmo, ya que las operaciones que aplicamos, en teoría, dependen del logaritmo del número en cuestión, o dicho de otra forma, del tamaño de la entrada. No obstante, en los resultados muchas de ellas tienen costo uniforme por trabajar con números de tamaño acotado (\texttt{unsigned long long int}) para simplificar la implementación.\Pa

		Sea $m = n$ y $tmp = 1$ \\
		\begin{pseudo}
			\WHILE{$m>0$} & \tOde{\log^3 n} \\
			\tab \IF{ $m$ es impar } & \tOde{\log m} \\
			\tab \tab $tmp \leftarrow tmp * b$ & \tOde{\log^{2}n} \\
			\tab \tab $tmp \leftarrow tmp\; mod\; n$ & \tOde{\log^{2}n} \\
			\tab $m \leftarrow \frac{m}{2}$ & \tOde{\log m} \\
			\tab $b \leftarrow b^2$ & \tOde{\log^{2} n} \\
			\tab $b \leftarrow b\; mod \; n$ & \tOde{\log^{2} n} \\
			\RET $tmp$
		\end{pseudo}
		
		\noindent\textbf{Cantidad de operaciones}\\

		\begin{center}
		\begin{tabular}{rlcl}
			1er&iteración & $\rightarrow$ & $m = n$ \\
			2da&iteración & $\rightarrow$ & $m = \frac{n}{2}$ \\
			3er&iteración & $\rightarrow$ & $m = \frac{n}{2^2}$ \\
			4ta&iteración & $\rightarrow$ & $m = \frac{n}{2^3}$ \\
			&\vdots&&\vdots \\
			$k$-ésima&iteración & $\rightarrow$ & $m = \frac{n}{2^{k-1}} = 1$  
		\end{tabular}
		\end{center}

		\noindent Como el algoritmo termina cuando $m=1$ tenemos, \\
		\begin{eqnarray*}
			\frac{n}{2^{k-1}}&=& 1 \\
			n &=& 2^{k-1} \\
			\log n &=& \log\; 2^{k-1} \\
			\log n &=& k-1 \Rightarrow k = \log( n )+1
		\end{eqnarray*}

		\noindent es decir, el algoritmo hace $\log(n)+1$ iteraciones, cada una con una cantidad constante de operaciones.

		Por lo tanto, la cantidad de operaciones que hace el algoritmo es del orden de $\log n$ y \Ode{\log n} $\subset$ \Ode{n}.

		\vspace{0.5cm}
		\noindent\textbf{Complejidad en el modelo logarítimico}\\

Como $m\leq n$ y la función logaritmo es estrictamente creciente, $\log m \leq \log n$. De la misma manera, $\log b$ y $\log tmp$ están acotados por $\log n$. Por lo tanto, la complejidad en el modelo logarítmico es: \\
\Ode{\log(n)*(\log(n)+\log^2(n)+\log^2(n)+\log(n)+\log^2(n)+\log^2(n))} $=$ \\
$=$ \Ode{\log(n)*(2\log(n)+4\log^2(n))} $=$ \Ode{2\log^2 n + 4\log^3 n} \\

Luego, por definición, 
			\begin{center}
				\Ode{2\log^2 n + 4\log^3 n} $=$ \Ode{max(2\log^2 n , 4\log^3 n)} $=$ \Ode{\log^3 n}.
			\end{center}

		Entonces, la complejidad del algoritmo resulta ser: \Ode{log^3 n}\VSP
		
		\noindent\textbf{En función del tamaño de la entrada}\\

			El tamaño de la entrada $t$ es $\log n$ ya que $b$ es acotado ($n=2^t$). Entonces la complejidad del algoritmo en función del tamaño de entrada es: \Ode{\log^3 2^t}=\Ode{t^3}. El algoritmo es logarítmico.

	\end{subsection}

	\begin{subsection}{Pruebas y Resultados}
		Para probar correctitud tenemos un generador de instancias random que nos devuelve dos archivos, en uno \texttt{test.in} la instancia elegida ($b$ $n$) y otro \texttt{test.out} el resultado ($b^n \mod n$). De esta forma, con una comparación de archivos (comando $diff\;test.out\;bn\_mod\_n.out$ ) podemos saber para cada instancia si el resultado obtenido por nuestro algoritmo se condice con el resultado correspondiente a esa instancia en el archivo \texttt{test.out}.
		
		Para llevar a cabo las pruebas de este algoritmo en cuanto a operaciones realizadas por instancia, y tiempo en segundos transcurridos, generamos números en forma aleatoria, con $b$ comprendido entre $0$ y $200$, y $n$ entre 1 y $10^7$.\\
		
		\gra{ej1_counts}\VSP
		
		En este gráfico se muestra la función que dada una instancia, nos devuelve la cantidad de operaciones realizadas. Se ve claramente cómo esta acotada por la función logaritmo, con $n$ como parámetro de entrada. Se pueden apreciar algunos casos en el fondo del gráfico que co\-rres\-pon\-den al caso trivial del ejercicio, en los que no hubo necesidad de iterar para llegar al resultado.

		\gra{ej1_time}\VSP

		En este otro gráfico se observa de forma muy parecida a la función que dada una instancia devuelve el tiempo transcurridos durante la ejecución del algoritmo. También se ve acotada por la función logaritmo con parámetro $n$. Como medir el tiempo no es una herramienta muy precisa, se pueden ver algunos {\em outliers} por encima de la curva del logaritmo. Aun así, los resultados coninciden con las predicciones en cuanto a la complejidad.\Pa

		En general, podemos ver que los resultados prácticos se mantienen cada vez más abajo de la complejidad teórica analizada (línea verde) y se corresponden en forma con la misma. Por lo que sería lógico creer que también lo va a hacer para mayores tamaños de entrada que los aquí representados.
		Por esto suponemos que es una buena cota para la complejidad del algoritmo.
	\end{subsection}

\end{section}

