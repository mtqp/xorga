\begin{section}{Problema 1}

	\textit{Dados $b,n \in \mathbb{N} $ calcular $b^n\; mod\; n$}

	\begin{subsection}{Explicación}

		Para la resolución del problema, usamos la técnica de $Divide$ $\&$ $Conquer$ para tratar de minimizar la cantidad
	de multiplicaciones. 

		\begin{subsubsection}{Análisis de complejidad}
		Esta vez, elegimos un modelo logarítmico para analizar el algoritmo, ya que las operaciones que aplicamos, 
		en teoría, dependen del logaritmo del número en cuestión, o dicho de otra forma, del tamaño de la entrada.
		No obstante, en los resultados muchas de ellas tienen costo uniforme por trabajar con números de tamaño acotado (\texttt{unsigned long long int}) para simplificar la implementación.

		Por esto tambien analizamos un poco la complejidad obtenida bajo el modelo uniforme. \\

		\begin{pseudo}
			\WHILE{$m>0$} \\
			\tab \IF{ $m$ es impar } \\
			\tab \tab $tmp \leftarrow tmp \cdot c$ \\
			\tab \tab $tmp \leftarrow tmp\; mod\; n$ \\
			\tab $m \leftarrow \frac{m}{2}$ \\
			\tab $c \leftarrow c^2$ \\
			\tab $c \leftarrow c\; mod\; n$
			
		\end{pseudo}

		\begin{tabular}{l l}
			1 iteración & $m = n$ \\
			2 iteración & $m = \frac{n}{2}$ \\
			3 iteración & $m = \frac{n}{2^2}$ \\
			4 iteración & $m = \frac{n}{2^3}$ \\
			k iteración & $m = \frac{n}{2^{k-1}} = 1$  
		\end{tabular}

		Como el algoritmo termina cuando $m=1$ entonces, \\
			$\frac{n}{2^{k-1}} = 1$ \\
			$n = 2^{k-1}$ \\
			$\log_2 n = \log_2 2^{k-1}$ \\
			$\log_2 n = k-1 \Rightarrow k = \log_2( n )+1 $ \\

		Entonces, el algoritmo hace $\log_2(n)+1$ iteraciones, cada una con una cantidad constante de operaciones. 
		Por lo tanto, la cantidad de operaciones que hace el algoritmo es del orden de $\log_2 n$.

		\end{subsubsection}
	\end{subsection}

	\begin{subsection}{Detalles de la implementación}
		El algoritmo primero calcula $c = b\;mod\;n$ para 

	\end{subsection}

	\begin{subsection}{Pruebas y Resultados}

	\end{subsection}

\end{section}

