\section{Discusi\'on}

El c\'odigo de este trabajo utiliza las funciones en lenguaje C del trabajo anterior y nuevas funciones en assembler. El trabajo en assembler esta separado en un archivo por filtro a aplicar y un  archivo para las macros en com\'un, utilizadas por los filtros.
	
A continuaci\'on se detallar\'a lo escrito y pensado para el algoritmo de Sobel, que es an\'alogo para el resto de los operadores, ya que con cambios m\'inimos se logr\'o implementarlos, salvo el Frei-Chen que ser\'a descripto mas adelante. \\

\subsection{Operadores}

Al comienzo se program\'o la funci\'on para el operador de Sobel ya que pod\'iamos comparar los resultados con el m\'etodo cvSobel de la librer\'ia. Una vez programado, se aprovech\'o lo escrito para implementar los otros operadores, realizando cambios m\'inimos al c\'odigo. Pasaremos entonces a detallar lo trabajado sobre el archivo asmSobel y luego los cambios hechos para llegar a la implementaci\'on de los otros archivos. \\

\noindent La operaci\'on consiste en un ciclo que comienza levantando los primeros 16 p\'ixels de la imagen. Para poder operar las 3 l\'ineas de la matr\'iz sin saturar antes de tiempo, se convierten los bytes (p\'ixeles) en words y las dos partes de 8 p\'ixels son almacenadas en registros distintos. Luego se procede a procesar cada parte. Para poder aplicar las l\'ineas a todos los p\'ixels obtenidos, la matr\'iz de Sobel fue extendida a 8 words, quedando cada l\'inea definida de la siguiente manera: \\

\code{
\begin{tabular}{r r r r r r r r r}
	Sobel X l\'inea 1&-1&0&1&-1&0&1&-1&0 \\
	Sobel X l\'inea 2&-2&0&2&-2&0&2&-2&0 \\ \\
	Sobel Y l\'inea 1&-1&-2&-1&-1&-2&-1&-1&-2 \\
	Sobel Y l\'inea 3&1&2&1&1&2&1&1&2 \\
\end{tabular}
} 

\vspace{0.5cm}

Para facilitar la comprensi\'on del algoritmo se utilizar\'an las siguientes representaciones: \\
\code{
\begin{tabular}{r c}
	L\'inea de la matr\'iz	&\reg{x}{y}{z}{x}{y}{z}{x}{y} \\ \\
	Primeros 8px de src	&\reg{a}{b}{c}{d}{e}{f}{g}{h} \\ \\
	\'Ultimos 8px de src	&\reg{i}{j}{k}{l}{m}{n}{o}{p} \\
\end{tabular} \\
} \\


Para procesar cada parte, se utiliz\'o el siguiente algoritmo:
\begin{itemize}

\item Desplaza el registro que contiene los p\'ixeles las veces necesarias para dejar el primer p\'ixel a calcular al principio. Por ejemplo, para procesar \code{a} y \code{d} no hace falta desplazar, pero para procesar \code{b} y \code{e} desplaza una vez hacia la derecha.

\item Multiplica los 8 p\'ixels correspondientes con la l\'inea de la matr\'iz que est\'a siendo procesada.\\*

\code{
\begin{center}
	\reg{a$*$x}{b$*$y}{c$*$z}{d$*$x}{e$*$y}{f$*$z}{g$*$x}{h$*$y}
\end{center}
}

\item Copia el resultado a un registro auxiliar y luego se al\'inean los p\'ixeles que ser\'an operados en breve.\\*

\code{
\begin{tabular}{r c}
	Temporal (1ro)	&\reg{a$*$x}{b$*$y}{c$*$z}{d$*$x}{e$*$y}{f$*$z}{g$*$x}{h$*$y} \\ \\
	Temporal (2do)	&\reg{b$*$y}{c$*$z}{d$*$x}{e$*$y}{f$*$z}{g$*$x}{h$*$y}{0} \\
\end{tabular} \\
}

\item Hace la suma entre los dos registros.\\*

\code{
\begin{tabular}{r c}
	Resultado Parcial &\reg{a$*$x+b$*$y}{-}{-}{d$*$x+e$*$z}{-}{-}{-}{-} \\ \\
\end{tabular} \\
}

\item Alinea al principio del registro temporal el tercer pixel. \\*

\code{
\begin{tabular}{r c}
	Temporal 	&\reg{c$*$z}{d$*$x}{e$*$y}{f$*$z}{g$*$x}{h$*$y}{0}{0} \\
\end{tabular} \\
}

\item Hace la suma entre los dos registros.\\*

\code{
\begin{tabular}{r c}
	Resultado Parcial &\reg{a$*$x+b$*$y+c$*$z}{-}{-}{d$*$x+e$*$z+f$*$z}{-}{-}{-}{-} \\ \\
\end{tabular} \\
}

\item Pone en 0 todas los bytes que contienen informaci\'on no \'util. \\*

\code{
\begin{tabular}{r c}
	Resultado Parcial &\reg{a$*$x+b$*$y+c$*$z}{0}{0}{d$*$x+e$*$z+f$*$z}{0}{0}{0}{0} \\ \\
\end{tabular} \\
}

\item Desplaza el resultado parcial a la ubicaci\'on del destino donde se tiene que acumular. Por ejemplo, si se procesaron \code{a} y \code{d}, desplaza una vez hacia la izquierda. \\

\item Suma el resultado parcial en el acumulador correspondiente. \\

\item Repite el procedimiento hasta calcular los 6 p\'ixeles centrales. \\*

\code{
\begin{tabular}{c}
\small
\reg{0}{a$*$x+b$*$y+c$*$z}{b$*$x+c$*$y+d$*$z}{c$*$x+d$*$y+e$*$z}{d$*$x+e$*$z+f$*$z}{e$*$x+f$*$y+g$*$z}{f$*$x+g$*$y+h$*$z}{0} \\ \\
\end{tabular} \\
}

\item Calcula por separado \code{g$*$x+h$*$y+i$*$z}, que hab\'ia quedado sin procesar

\item Calcula por separado \code{h$*$x+i$*$y+j$*$z}

\item Repite el procesamiento (que se hizo con la parte baja) con la parte alta

\item Repite el procesamiento de la l\'inea con las 2 siguientes l\'ineas, cambiando a las l\'ineas de matr\'iz correspondientes

\item Empaqueta a bytes los acumuladores

\item Copia el resultado al destino

\item Avanza 14 p\'ixels

\item Repite todo el procedimiento hasta terminar la imagen

\end{itemize}


\subsection{Medici\'on de Performance}

La siguiente tabla muestra la cantidad de ciclos m\'inima y promedio de cada implementaci\'on de los filtros, obtenidos de una muestra de
1000 ejecuciones de cada uno sobre la imagen de prueba \code{lena.bmp}:
\begin{center}
\begin{tabular}{|l|r|r|}
\hline
\multirow{2}{*}{Implementaci\'on}&\multicolumn{2}{|c|}{Ciclos de reloj} \\
\cline{2-3}
&M\'inimo	&	Promedio \\
\hline
\multicolumn{3}{|c|}{Sobel}\\
\hline
Assembler	&	58.720.540	&	60.010.675 \\
\hline
C		&	393.586.848	&	416.838.429 \\
\hline
OpenCv		&	9.338.589	& 	9.797.886 \\
\hline
SSE		&	7.845.504 	&	7.991.641 \\
\hline
\multicolumn{3}{|c|}{Roberts}\\
\hline
Assembler	&	34.714.238	&	42.746.947 \\
\hline
C		&	320.676.453	&	336.808.912 \\
\hline
SSE		&	1.749.548	&	1.915.205 \\
\hline
\multicolumn{3}{|c|}{Prewitt}\\
\hline
Assembler	&	60.428.397	&	63.629.648 \\
\hline
C		&	634.337.521	&	664.132.063 \\
\hline
SSE		&	8.024.988		&	8.276.912 \\
\hline

\end{tabular}
\end{center}



\pagebreak
