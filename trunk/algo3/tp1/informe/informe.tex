\documentclass[12pt,titlepage]{article}
\usepackage[spanish]{babel}
\usepackage[utf8]{inputenc}
\usepackage{amsfonts}
\usepackage{amsmath}
\usepackage{color}
\usepackage{graphicx} % para insertar imagenes


\newcommand{\func}[2]{\texttt{#1}(#2)\\}
\newcommand{\tab}{\hspace*{2em}}
\newcommand{\FOR}{\textbf{for }}
\newcommand{\TO}{\textbf{ to }}
\newcommand{\IF}{\textbf{if }}
\newcommand{\WHILE}{\textbf{while }}
\newcommand{\THEN}{\textbf{then }}
\newcommand{\ELSE}{\textbf{else }}
\newcommand{\RET}{\textbf{return }}
\newcommand{\MOD}{\textbf{ \% }}
\newcommand{\OR}{\textbf{ or }}
\newcommand{\tOde}[1]{\tab \small{O($#1$)}}
\newcommand{\Ode}[1]{O($#1$)}
\newcommand{\Thetade}[1]{{\small$\Theta$($#1$)}}
\newcommand{\Omegade}[1]{{\small$\Omega$($#1$)}}
\newcommand{\VSP}{\vspace*{3em}}
\newcommand{\Pa}{\vspace{5mm}}
\newenvironment{pseudo}{\begin{tabular}{ll}}{\end{tabular}\VSP}

\newcommand{\gra}[1]{\noindent\includegraphics[scale=.60]{#1}\\}

\title{{\sc\normalsize Algoritmos y estructuras de datos III}\\{\bf Trabajo Práctico Nº1}}
\author{\begin{tabular}{lcr}
Carla Livorno & 424/08 & carlalivorno@hotmail.com\\
Daniel Grosso & 694/08 & dgrosso@gmail.com\\
Diego Raffo & 423/08 & enanodr@hotmail.com \\
Mariano De Sousa Bispo & 389/08 & marian\_sabianaa@hotmail.com \\
\end{tabular}}
\date{\VSP \normalsize{Abril 2010}}
%\date{}
\begin{document}
\begin{titlepage}
\maketitle
\end{titlepage}
\tableofcontents
\newpage

	\begin{section}*{Introducción}	\addcontentsline{toc}{section}{Introducción}
	Este trabajo tiene como objetivo la aplicación de diferentes técnicas algorítmicas para la resolución de tres problemas particulares, el cálculo de complejidad teórica en el peor caso de cada algoritmo implementado, y la posterior verificación empírica.
	
	El lenguaje utilizado para implementar los algoritmos de todos los problemas fue C/C++
	\end{section}

	\begin{section}{Problema 1}

	\textit{enunciado}

	\begin{subsection}{Explicación}

		La resolución del problema consiste en considerar cada uno de los elementos de la secuencia dada como
		posible máximo de una subsecuencia (al que nos referiremos como ``pico"), tal que sea unimodal. Cada 
		subsecuencia unimodal tiene longitud máxima, es decir, contiene la subsecuencia creciente hasta el pico
		y la subsecuencia decreciente desde el pico, ambas con la mayor cantidad de elementos posibles. Al no
		necesitar explicitar los elementos de la secuencia, la única información que aporta a la solución del
		problema es la longitud de cada subsecuencia. De esta manera, para determinar la mínima cantidad de
		elementos que se deben eliminar para transformar la secuencia dada en unimodal, basta con conocer
		la diferencia entre la longitud de la secuencia original y el máximo de las longitudes de cada
		subsecuencia unimodal. 		

	\end{subsection}


	\begin{subsection}{Detalles de la implementación}
		
		Para determinar la máxima longitud de la subsecuencia creciente y decreciente hasta cada elemento,
		utilizamos la técnica de programación dinámica.
		
		Sea la secuencia dada $S=[s_1,s_2,...,s_n]$ y $C=[c_1,c_2,...,c_n]$ \\
		con $c_i=\displaystyle\max_{0<j<i}\{ c_j / s_j < s_i \}+1$ ($\forall\;i\in\mathbb{N}, 0<i\leq n$), es 
		decir, para cada $i$ tenemos en $c_i$ la máxima longitud de la subsecuencia creciente que incluye a
		$s_i$. Análogamente, se define $D$ como la secuencia que contiene las máximas longitudes de las
		subsecuencias decrecientes de $S$.
		
		El algoritmo que calcula la solución implementa tanto $C$ como $D$.	
		Para obtener la máxima longitud de la subsecuencia creciente hasta el índice $i$, itera
		por todos los índices $j<i$ en $C$, buscando el máximo valor entre los $c_j$ tales que $s_i$ sea mayor
		que	$s_j$. 
		Esto asegura que en la posición $c_i$ está la máxima longitud de la subsecuencia creciente
		que incluye a $s_i$ ya que, si existiese otra subsecuencia de mayor longitud a la que se pueda 
		agregar $s_i$, se podría agregar el $s_i$ a esa secuencia y así obtener una con más cantidad de 
		elementos, siendo el valor de $c_i$ la longitud de dicha secuencia más 1. Para calcular $D$,
		invertimos $S$ y aplicamos el mismo	procedimiento que para $C$, quedando en $d_i$ la máxima longitud
		de la subsecuencia decreciente que incluye a $s_{n-i}$.

		Luego de calcular $C$ y $D$ el algoritmo determina la mínima cantidad de elementos a ser eliminados de
		la secuencia tales que el resto de los elementos forman una secuencia unimodal, de la siguiente forma:
		
		\vspace{0.5cm}
		\begin{pseudo}
			\func{secuencia\_unimodal}{$secuencia,C,D$}
			\tab $max\_long \leftarrow 0$ \\
			\tab \FOR $i=1$ \TO $n$ \\
			\tab \tab $long\_secuencia\_unimodal\leftarrow C[i]+D[n-i]-1$ \\
			\tab \tab \IF $ long\_secuencia\_unimodal >  max\_long$ \\
			\tab \tab \tab $max\_long \leftarrow long\_secuencia\_unimodal$ \\
			\tab \RET longitud($secuencia$)-$max\_long$
		\end{pseudo}

	\end{subsection}


	\begin{subsection}{Análisis de complejidad}
	Como hemos visto anteriormente, nuestro algoritmo utiliza la técnica de programación dinámica. Esta consta de reutilizar información previa para llegar al resultado final. En el análisis de la complejidad veremos como la aplicación de la técnica previamente mencionada, ha permitido conseguir un algoritmo polinomial.\\
	
	El algoritmo $long\_max\_creciente$ que calcula para cada elemento $i$ de la secuencia original la máxima longitud de la secuencia creciente y decreciente que lo incluye, itera por todos los índices $j<i$ ($\forall\; 0 \leq i < n$).\\

	La complejidad de dicho algoritmo viene dada por: $\sum_{i=0}^{n-1} i = \frac{n*(n-1)}{2} = ESTA SUMATORIA DA DEL OOOODEN DE ene cuadrado, OJO AHI LA PIPETUa... n^2$\\

	Sea $n$ la longitud de la secuencia dada, $max\_long\_creciente$ la secuencia que guarda en cada posición la máxima longitud de la subsecuencia creciente.\\

	\begin{pseudo}
		\func{long\_max\_creciente}{$secuencia, max\_long\_creciente$}
		\tab $max\_long\_creciente[0] \leftarrow 0$ \tab \tab \Ode{1} \\
		\tab \FOR $i=1$ \TO $n$ \\
		\tab \tab $max\_long \leftarrow 0$ \tab \tab \tab \tab \Ode{1} \\
		\tab \tab \FOR $j=i-1$ \TO $0$ \tab \Ode{i}\\
		\tab \tab \tab \IF $ secuencia[j] <  secuencia[i] \AND max\_long\_creciente [j] > max\_long)$ \Ode{1} \\
		\tab \tab \tab \tab  $max\_long \leftarrow max\_long\_creciente[j] $ \tab \Ode{1} \\
		\tab $max\_long\_creciente[i] \leftarrow max\_long+1$ \tab \tab \tab \Ode{1} \\
	\end{pseudo}

	Podemos ver dentro de cada ciclo ($for$) que todas las asignaciones tienen costo constante, así como también la guardar del ($if$). De esta manera podemos concluir que por cada iteración del ($for$) anidado tenemos en el peor caso el costo de la guarda del if, más el costo de la asignación adentro del mismo, con costo constante.\\
	El ciclo anidado iterará para cada $i$, $i-1$ veces, siendo los valores posibles de $i$ desde $1$ hasta $n$.\\
	La sumatoria que fue descripta anteriormente, se coindice entonces con el costo de la función $long\_max\_creciente$. La cantidad de operaciones que realiza este algoritmo es \Ode{n^2}.\\
	
	El algoritmo encargardo de devolver el resultado del problema se llama $secuencia\_unimodal$ la cual utiliza el algoritmo descripto anteriormente y realizar algunas otras operaciones que se detallarán a continuación para concluir con el análisis de complejidad.\\
	
	\end{subsection}


	\begin{subsection}{Pruebas y Resultados}

	\end{subsection}

\end{section}



	\newpage

	\begin{section}{Problema 2}
	
	\begin{subsection}{Explicación}

		\begin{subsubsection}{Análisis de complejidad}

		\end{subsubsection}
	\end{subsection}

	\begin{subsection}{Detalles de la implementación}
	
	\end{subsection}

	\begin{subsection}{Pruebas y Resultados}
		
	\end{subsection}
\end{section}



	\newpage

	\begin{section}{Problema 3}

	\textit{enunciado}

	\begin{subsection}{Explicación}
		Como Bernardo puede pasar más de una vez por cada habitación el algoritmo restringe el acceso a cada una de estas, podrá acceder a cada habitación tantas veces como pasillos llegan a la habitación.
		Bernardo recorre las habitaciones que tengan conexión con la habitación donde se encuentra, siempre y cuando pueda entrar, ya sea porque tiene la llave o la habitación no tiene puerta y además, todavía tiene acceso a la habitación a entrar. Su procedimiento continúa hasta que:
		\begin{itemize}
			\item Llega a la habitación $n$, por lo tanto encontró la salida.
			\item Se le terminan los accesos a las habitaciones vecinas y por lo tanto no puede escapar.
		\end{itemize}

	\end{subsection}

	\begin{subsection}{Detalles de la implementación}
		Modelamos este problema con un grafo donde los vértices corresponden a las habitaciones (puede ser una habitación con una llave dentro, con una puerta o sin puerta ni llave) y las aritas a los pasillos.\\	

		Generamos la matriz de adyacencia del grafo (de $n\times n$ donde $n$ es la cantidad de habitaciones donde cada posición $(i,j)$ de la matriz contiene un $uno$ si la habitación $i$ está conectada con la habitación $j$. y $cero$ en caso contrario). Nos referiremos a la matriz como $conexiones$.\\

		Para toda habitación que tenga puerta existe una llave. Poseer esta llave implica tener la posibilidad de acceder a la habitación. Podemos abstraernos del problema de Bernardo y considerar a las llaves como valores booleanos en un arreglo que nos dice para cada vértice, si este es accesible o no. Tenemos entonces, un arreglo de tipo bool ($tengo\_llave$) de tamaño $n$ donde cada vértice, representado por el índice de dicho arreglo, nos dice si es accesible o no.\\

		Por otro lado, tenemos un arreglo $puertas$ de tamaño $n$ (donde $n$ es la cantidad de vértices del grafo), donde cada posición, si corresponde a una habitación con llave, tiene el vértice al cual habilita el acceso. En caso contrario, el arreglo contiene el valor $0$. El valor es cero, PORQUE NOS LA RE BANCAMOS MODIFICANDO EL VALOR DE LA ROOM DE ENTRADA... ACLARAR BABY\\

		Para la resolución del problema recorrimos el grafo de forma ordenada por niveles. Para esto hicimos una modificación al algoritmo $Breadth\; First\; Search$. La modificación consiste en visitar por niveles hasta que algún vértice no sea accesible  asdpogkadpofgadpfo 




La modificación consiste en no visitar indistitamente los vértices todavía no visitados (con la única condición de que sean adyacentes al vértice actual), sino que visitar aquellos vértices que además de ser adyacentes, $tengo\_llave$ del vértice al que quiero acceder es verdadero y todavía me quedan accesos al vértice (el arreglo accesos posee un valor mayor a cero para el mismo).\\

		El objetivo del $bfs$ es llegar desde el primer vértice al último.\\

		El $bfs$ encola los vértices con un cierto orden de prioridad. Los vértices que no han sido visitados tienen mayor prioridad frente a los ya visitados (es decir, se agrega primero todos los no visitados, seguidos de todos los visitados en orden indistinto entre ellos). Para cada vértice que se encola, su cantidad de accesos disminuye en uno y como ya mencionamos, no se puede utilizar un vértice con cantidad de accesos en cero. Como la cantidad de accesos a cada vértice es limitada, queremos explorar caminos todavía no vistos porque estos pueden concedernos el acceso a algún vértice que continúa algún camino ya visitado. Si encolaramos los vértices con prioridad inversa podría pasar que agotemos la cantidad de accesos de los vértices, encontrando luego el acceso (llave) a nuevos vértices que ya no podemos acceder porque para llegar a él, se necesita pasar por alguno que fue bloqueado.\\

		El siguiente pseudocódigo refleja el comportamiento previamente descripto:\\

		\begin{pseudo}
		\func {prision}{$conexiones, tengo\_llave, puertas, accesos, n$}
		\tab $llegue \leftarrow false$\\
		\tab $cola\; q$\\
		\tab $visitados[0..n] \leftarrow false$\\ 
		\tab $encolar(q,0)$\\
		\tab $accesos[0] \leftarrow accesos[0]-1$\\
		\tab $visitados[0] \leftarrow true$\\
		\tab \WHILE $!esVacia(q) \AND !llegue$\\
		\tab \tab $actual \leftarrow primero(q)$\\
		\tab \tab $desencolar(q)$\\
		\tab \tab \FOR $ i\leftarrow0 \TO n \AND !llegue$\\
		\tab \tab \tab \IF $es\_adyacente(actual,i) \AND tengo\_llave[i] \AND accesos[i]>0$\\
		\tab \tab \tab \tab $tengo\_llave[puertas[i]] \leftarrow true$\\
		\tab \tab \tab \tab $accesos[i] \leftarrow accesos[i]-1$\\
		\tab \tab \tab \tab $visitados[i] \leftarrow true$\\
		\tab \tab \tab \tab $llegue \leftarrow (i == n-1)$\\
		\tab \tab $encolar\_mayor\_prioridad(q, adyacentes(actual))$\\
		\tab \tab $encolar\_menor\_prioridad(q, adyacentes(actual))$\\
		\tab \RET $llegue$\\
		\end{pseudo}

		El resultado final viene dado de la variable $llegue$. Es inicializada en falso y se setea en verdadero si sólo si en algún momento $i == n-1$. Si esto ocurre, podemos concluir que pudimos llegar al vértice objetivo.\\
	\end{subsection}


	\begin{subsection}{Análisis de complejidad}
		El \textbf{while} se ejecuta a lo sumo $n$ veces
		
	\end{subsection}


	\begin{subsection}{Pruebas y Resultados}

	\end{subsection}

\end{section}


	
	\newpage
	%	\begin{subsection}{Anexo}
		La siguiente tabla representa la correspondencia entre las variables de entrada ($b$ y $n$) en 
		cada iteración del algoritmo implementado:

		\vspace{0.5cm}
		\begin{center}
		\begin{tabular}{|l|c|c|c|c|c|}
			\hline
			iteración   & $1$ & $2$           & $3$             & ... & $k$ \\
			\hline
			$n$         & $n$ & $\frac{n}{2}$ & $\frac{n}{2^2}$ & ... & $\frac{n}{2^{k-1}}$ \\
			\hline
			$b$         & $b$ & $b^2$         & $b^4$           & ... & $b^{2^{k-1}}$ \\
			\hline
		\end{tabular}
		\end{center}

		\vspace{0.5cm}
		\noindent Sean \\
		\indent
		\begin{tabular}{lp{6cm}}
			$A_k = \frac{n}{2^{k-1}}$ & la sucesión con los valores de $n$ en la iteración $k$, y \\
			$Z_k = impar(A_k) * b^{2^{k-1}} + par( A_k )$ $^{[1]}$ & la sucesión que tiene los valores de $b$ corres\-pondientes a los $n$ impares y $1$ en los pares
		\end{tabular} \\
		\vspace{0.2cm}
		entonces el cálculo hecho por el algoritmo está dado por 
		$$\displaystyle\prod_{i=1}^k Z_i = b^n$$


		\vspace{0.5cm}
		\noindent{\footnotesize [1] $par(x)=1-impar(x)$\\ $impar$ se define como:
		\begin{displaymath}
			impar(x)=\left\{
			\begin{array}{ll}
				1 & $si $x$ es impar$ \\
				0 & $sino$
			\end{array}\right.
		\end{displaymath}
		} \\
		
	\end{subsection}

	
	\newpage
	
	\begin{section}{Compilación y ejecución de los programas}
	Para compilar los programas se puede usar el comando \texttt{make} (Requiere el compilador \texttt{g++}).
	Se pueden correr los programas de cada ejercicio ejecutando \texttt{./bn\_mod\_n}, \texttt{./ronda\_de\_amigas} y \texttt{./programadores} respectivamente.
		
	Los programas leen la entrada de stdin y escriben la respuesta en stdout. 		Para leer la entrada de un archivo \texttt{Tp1EjX.in} y escribir la respuesta en un archivo \texttt{Tp1EjX.out} ses puede usar:\\ \texttt{./(ejecutable) < Tp1EjX.in > Tp1EjX.out}
		
	Para medir los tiempos de ejecución: \texttt{./(ejecutable) time}. Devuelve para cada instancia el tamaño seguido de la cantidad de ciclos de reloj consumidos.

	Para contar la cantidad de operaciones: \texttt{./(ejecutable) count}. Devuelve para cada instancia el tamaño seguido de la cantidad de operaciones de cada instancia.
	\end{section}
	
	\begin{section}{Conclusiones}
		Pudimos implementar las soluciones a los problemas propuestos utilizando diversas técnicas algorítmicas. En todos los casos las mediciones de tiempo y cantidad de operaciones se correspondieron con las complejidades teóricas calculadas. Esto indica que los modelos elegidos fueron adecuados.

		En el problema $dos$ nos encontramos con un algoritmo de complejidad exponencial que a partir de cierto tamaño de entrada 'relativamente chico' se torna inutil, ya que la ejecución para dichas instancias es extremadamente lenta y en la practica muchas veces esto no es aceptable. Esto significa que el problema no esta 'bien' resuelto para tamaños de entrada no necesariamente grandes.
	\end{section}

\end{document}
