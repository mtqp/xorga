\begin{section}{Problema 1}

	Dados $b,n \in \mathbb{N} $ calcular $b^n\; mod\; n$

	\begin{subsection}{Explicación}

		Para la resolución del problema, usamos la técnica de $Divide$ $\&$ $Conquer$ para tratar de minimizar la cantidad
	de multiplicaciones. 

		\begin{subsubsection}{Análisis de complejidad}
		Esta vez, elegimos un modelo logarítmico para analizar el algoritmo, ya que las operaciones que aplicamos, 
		en teoría, dependen del logaritmo del número en cuestión, o dicho de otra forma, del tamaño de la entrada.
		No obstante, en los resultados muchas de ellas tienen costo uniforme por trabajar con números de tamaño acotado (\texttt{unsigned long long int}) para simplificar la implementación.

		Por esto tambien analizamos un poco la complejidad obtenida bajo el modelo uniforme. \\

		\end{subsubsection}
	\end{subsection}

	\begin{subsection}{Detalles de la implementación}
		El algoritmo primero calcula $c = b\;mod\;n$ para 

	\end{subsection}

	\begin{subsection}{Pruebas y Resultados}

	\end{subsection}

\end{section}

