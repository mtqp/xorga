\begin{section}{Ejercicio 3}
	
	\begin{subsection}{Explicación}
El ejercicio planteaba el problema de obtener la mayor cantidad de programadores que están simultáneamente dentro de una empresa de software dado los horarios de ingreso y egreso de cada programador.
Vamos recorriendo la lista de ingresos y egresos, viendo la cantidad de programadores que hay en cada horario registrado. Si cuando egresa una persona, la cantidad de programadores hasta ese momento era mayor que la cantidad máxima registrada previamente, se establece la nueva cantidad máxima. Al terminar de recorrer la lista de ingresos, 


		\begin{subsubsection}{Análisis de complejidad}
		\end{subsubsection}

	\end{subsection}

	\begin{subsection}{Detalles de la implementación}
	\end{subsection}

	\begin{subsection}{Pruebas y Resultados}			
	\end{subsection}

\end{section}







