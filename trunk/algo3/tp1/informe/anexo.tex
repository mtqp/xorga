	\begin{subsection}{Anexo}
		\begin{subsubsection}{Correctitud ejercicio 1}
		La siguiente tabla representa la correspondencia entre las variables de entrada ($b$ y $n$) en 
		cada iteración del algoritmo implementado:

		\vspace{0.5cm}
		\begin{center}
		\begin{tabular}{|l|c|c|c|c|c|}
			\hline
			iteración   & $1$ & $2$           & $3$             & ... & $k$ \\
			\hline
			$n$         & $n$ & $\frac{n}{2}$ & $\frac{n}{2^2}$ & ... & $\frac{n}{2^{k-1}}$ \\
			\hline
			$b$         & $b$ & $b^2$         & $b^4$           & ... & $b^{2^{k-1}}$ \\
			\hline
		\end{tabular}
		\end{center}

		\vspace{0.5cm}
		\noindent Sean \\
		\indent
		\begin{tabular}{lp{6cm}}
			$A_k = \frac{n}{2^{k-1}}$ & la sucesión con los valores de $n$ en la iteración $k$, y \\
			$Z_k = impar(A_k) * b^{2^{k-1}} + par( A_k )$ $^{[1]}$ & la sucesión que tiene los valores de $b$ corres\-pondientes a los $n$ impares y $1$ en los pares
		\end{tabular} \\
		\vspace{0.2cm}
		entonces el cálculo hecho por el algoritmo está dado por 
		$$\displaystyle\prod_{i=1}^k Z_i = b^n$$

		Luego de cada multiplicación toma módulo $n$ ya que
			$$b^k * b^{n-k}\; mod\;n = ((b^k\;mod\;n)*(b^{n-k}\;mod\;n))\;mod\;n\;\forall\;k\leq n ^{[2]}$$ 

		De esta manera, incluso si el cálculo de $b^n$ es un número tan grande que no entra en el 
		tamaño de la variable, se va a poder realizar sin problemas (suponiendo que $(n-1)^2$ entra 
		en una variable).


		\vspace{0.5cm}
		\noindent{\footnotesize [1] $par(x)=1-impar(x)$\\ $impar$ se define como:
		\begin{displaymath}
			impar(x)=\left\{
			\begin{array}{ll}
				1 & $si $x$ es impar$ \\
				0 & $sino$
			\end{array}\right.
		\end{displaymath}
		} \\
		\vspace{0.5cm}
		{\footnotesize [2] Por propiedades del módulo $x*y\; mod \; z = ((x\; mod\; z)*(y\; mod\; z))\;mod\; z$ } \\
		\end{subsubsection}
		\begin{subsubsection}{Resultado instancias que nunca entran al caso condicional}
		\end{subsubsection}
	\end{subsection}
