\documentclass[12pt,titlepage]{article}
\usepackage[spanish]{babel}
\usepackage[utf8]{inputenc}
\usepackage{amsfonts}
\usepackage{amsmath}
\usepackage{amssymb}
\usepackage{color}
\usepackage{graphicx} % para insertar imagenes
\usepackage{verbatim}


\newcommand{\func}[2]{\texttt{#1}(#2)\\}
\newcommand{\tab}{\hspace*{2em}}
\newcommand{\FOR}{\textbf{for }}
\newcommand{\TO}{\textbf{ to }}
\newcommand{\IF}{\textbf{if }}
\newcommand{\WHILE}{\textbf{while }}
\newcommand{\THEN}{\textbf{then }}
\newcommand{\ELSE}{\textbf{else }}
\newcommand{\RET}{\textbf{return }}
\newcommand{\MOD}{\textbf{ \% }}
\newcommand{\OR}{\textbf{ or }}
\newcommand{\AND}{\textbf{ and }}
\newcommand{\tOde}[1]{\tab \small{O($#1$)}}
\newcommand{\Ode}[1]{O($#1$)}
\newcommand{\Thetade}[1]{{\small$\Theta$($#1$)}}
\newcommand{\Omegade}[1]{{\small$\Omega$($#1$)}}
\newcommand{\VSP}{\vspace*{3em}}
\newcommand{\Pa}{\vspace{5mm}}
\newenvironment{pseudo}{\begin{tabular}{ll}}{\end{tabular}\VSP}

\newcommand{\gra}[1]{\noindent\includegraphics[scale=.60]{#1}\\}

\title{{\sc\normalsize Algoritmos y estructuras de datos III}\\{\bf Trabajo Práctico Nº1}}
\author{\begin{tabular}{lcr}
Carla Livorno & 424/08 & carlalivorno@hotmail.com\\
Daniel Grosso & 694/08 & dgrosso@gmail.com\\
Diego Raffo & 423/08 & enanodr@hotmail.com \\
Mariano De Sousa Bispo & 389/08 & marian\_sabianaa@hotmail.com \\
\end{tabular}}
\date{\VSP \normalsize{Abril 2010}}
%\date{}
\begin{document}
\begin{titlepage}
\maketitle
\end{titlepage}
\tableofcontents
\newpage

	\begin{section}*{Introducción}	\addcontentsline{toc}{section}{Introducción}
	Este trabajo tiene como objetivo la aplicación de diferentes técnicas algorítmicas para la resolución de tres problemas particulares, el cálculo de complejidad teórica en el peor caso de cada algoritmo implementado, y la posterior verificación empírica.
	
	El lenguaje utilizado para implementar los algoritmos de todos los problemas fue C/C++
	\end{section}

	\begin{section}{Problema 1}

	\textit{Sea $s = (s_1,s_2,..,s_n)$ una secuencia de números enteros. Determinar la mínima cantidad de elementos de $s$ tales que al ser eliminados de la secuencia, el resto de los elementos forman una secuencia unimodal.}

	\begin{subsection}{Explicación}

		La resolución del problema consiste en considerar cada uno de los elementos de la secuencia dada como
		posible máximo de una subsecuencia (al que nos referiremos como ``pico"), tal que sea unimodal. Cada 
		subsecuencia unimodal tiene longitud máxima, es decir, contiene la subsecuencia creciente hasta el pico
		y la subsecuencia decreciente desde el pico, ambas con la mayor cantidad de elementos posibles. Al no
		necesitar explicitar los elementos de la secuencia, la única información que aporta a la solución del
		problema es la longitud de cada subsecuencia. De esta manera, para determinar la mínima cantidad de
		elementos que se deben eliminar para transformar la secuencia dada en unimodal, basta con conocer
		la diferencia entre la longitud de la secuencia original y el máximo de las longitudes de cada
		subsecuencia unimodal. 		

	\end{subsection}


	\begin{subsection}{Detalles de la implementación}
		
		Para determinar la máxima longitud de la subsecuencia creciente y decreciente hasta cada elemento,
		utilizamos la técnica de programación dinámica.
		
		Sea la secuencia dada $S=[s_1,s_2,...,s_n]$ y $C=[c_1,c_2,...,c_n]$ \\
		con $c_i=\displaystyle\max_{0<j<i}\{ c_j / s_j < s_i \}+1$ ($\forall\;i\in\mathbb{N}, 0<i\leq n$), es 
		decir, para cada $i$ tenemos en $c_i$ la máxima longitud de la subsecuencia creciente que incluye a
		$s_i$. Análogamente, se define $D$ como la secuencia que contiene las máximas longitudes de las
		subsecuencias decrecientes de $S$.
		
		El algoritmo que calcula la solución implementa tanto $C$ como $D$.	
		Para obtener la máxima longitud de la subsecuencia creciente hasta el índice $i$, itera
		por todos los índices $j<i$ en $C$, buscando el máximo valor entre los $c_j$ tales que $s_i$ sea mayor
		que	$s_j$. 
		Esto asegura que en la posición $c_i$ está la máxima longitud de la subsecuencia creciente
		que incluye a $s_i$ ya que, si existiese otra subsecuencia de mayor longitud a la que se pueda 
		agregar $s_i$, se podría agregar el $s_i$ a esa secuencia y así obtener una con más cantidad de 
		elementos, siendo el valor de $c_i$ la longitud de dicha secuencia más 1. Para calcular $D$,
		invertimos $S$ y aplicamos el mismo	procedimiento que para $C$, quedando en $d_i$ la máxima longitud
		de la subsecuencia decreciente que incluye a $s_{n-i}$.

		Luego de calcular $C$ y $D$ el algoritmo determina la mínima cantidad de elementos a ser eliminados de
		la secuencia tales que el resto de los elementos forman una secuencia unimodal, de la siguiente forma:
		
		\vspace{0.5cm}
		\begin{pseudo}
			\func{secuencia\_unimodal}{$secuencia,C,D$}
			\tab $max\_long \leftarrow 0$ \\
			\tab \FOR $i=1$ \TO $n$ \\
			\tab \tab $long\_secuencia\_unimodal\leftarrow C[i]+D[n-i]-1$ \\
			\tab \tab \IF $ long\_secuencia\_unimodal >  max\_long$ \\
			\tab \tab \tab $max\_long \leftarrow long\_secuencia\_unimodal$ \\
			\tab \RET longitud($secuencia$)-$max\_long$
		\end{pseudo}

	\end{subsection}


	\begin{subsection}{Análisis de complejidad}
		Como hemos visto anteriormente, nuestro algoritmo utiliza la técnica de programación dinámica. Esta consta de 
		reutilizar información previa para llegar al resultado final. En el análisis de la complejidad veremos como la
		aplicación de la técnica previamente mencionada, ha permitido conseguir un algoritmo polinomial.
	
		El algoritmo $long\_max\_creciente$ utiliza la técnica de programación dinámica, calculando para cada elemento 
		$i$ de la secuencia original la máxima longitud de la subsecuencia creciente y decreciente que lo incluye, iterando 
		por todos los índices $j<i$ ($\forall\; 0 \leq i < n$), sabiendo que para cada $j$ ya esta calculada la longitud de 
		la subsecuencia creciente más larga que lo incluye.\\

		Sea $n$ la longitud de la secuencia dada, $max\_long\_creciente$ la secuencia que guarda en cada posición la máxima 
		longitud de la subsecuencia creciente.\\

		\begin{pseudo}
			\func{long\_max\_creciente}{$secuencia, max\_long\_creciente$}
			\tab $max\_long\_creciente[0] \leftarrow 0$ 		\Ode{1} \\
			\tab \FOR $i=1$ \TO $n$ \\
			\tab \tab $max\_long \leftarrow 0$ 			\Ode{1} \\
			\tab \tab \FOR $j=i-1$ \TO $0$ 				\Ode{i}\\
			\tab \tab \tab \IF $ secuencia[j] <  secuencia[i] \AND max\_long\_creciente [j] > max\_long)$ \Ode{1} \\
			\tab \tab \tab \tab  $max\_long \leftarrow max\_long\_creciente[j] $ 		\Ode{1} \\
			\tab \tab $max\_long\_creciente[i] \leftarrow max\_long+1$ 			\Ode{1} \\
		\end{pseudo}

		Podemos ver dentro de cada ciclo ($for$) que todas las asignaciones tienen costo constante, así como también la guardar 
		del ($if$). De esta manera podemos concluir que por cada iteración del ($for$) anidado tenemos en el peor caso el costo 
		de la guarda del if, más el costo de la asignación adentro del mismo, con costo constante.

		El ciclo anidado iterará para cada $i$, $i-1$ veces, siendo los valores posibles de $i$ desde $1$ hasta $n$.

		La complejidad de dicho algoritmo viene dada por: $\sum_{i=0}^{n-1} i = \frac{n*(n-1)}{2}$

		Esta sumatoria, se coindice entonces con el costo de la función $long\_max\_creciente$. La cantidad de operaciones que 
		realiza este algoritmo es \Ode{n^2}.\\
	
		El algoritmo encargardo de devolver el resultado del problema es $secuencia\_unimodal$, el cual utiliza el algoritmo 
		descripto anteriormente y realizar algunas otras operaciones que se detallarán a continuación para concluir con el análisis 
		de complejidad.\\
	
		\vspace{0.5cm}
		\begin{pseudo}
			\func{secuencia\_unimodal}{$secuencia, n$}
			\tab $C[n]$\\
			\tab $D[n]$\\
			\tab $R \leftarrow reverso(secuencia)$ 		\Ode{n} \\
			\tab $long\_max\_creciente(secuencia,C)$ 	\Ode{n^2}\\
			\tab $long\_max\_creciente(R,D)$		\Ode{n^2}\\
			\tab $max\_long \leftarrow 0$			\Ode{1}\\
			\tab \FOR $i=1$ \TO $n$				\Ode{n}\\
			\tab \tab $long\_secuencia\_unimodal\leftarrow C[i]+D[n-i]-1$ 	\Ode{1}\\
			\tab \tab \IF $ long\_secuencia\_unimodal >  max\_long$ 	\Ode{1}\\
			\tab \tab \tab $max\_long \leftarrow long\_secuencia\_unimodal$ \Ode{1}\\
			\tab \RET longitud($secuencia$)-$max\_long$ 			\Ode{1}\\
		\end{pseudo}
		
		La función reverso que toma una secuencia y devuelve otra con los elementos en orden inverso tiene costo lineal, en función 
		de la cantidad de elementos, ya que itera una vez por la secuencia original ($desde i=0 hasta n-1$), guardando cada valor en 
		la posición $n-1-i$ de la secuencia resultante. Cabe destacar que el costo de la asignación es constante.

		El ciclo perteneciente a $secuencia\_unimodal$ ($for, linea 7$), itera $n$ veces, siendo en el peor caso el costo de cada una, 
		constante (SE PODIRA PONER TODOS LOS O DEL MUNDO).

		Por lo tanto, el costo del algoritmo $secuencia\_unimodal$ es: \Ode{2*n + 2*n^2} que por definición es 
		\Ode{max(2*n, 2*n^2)} = \Ode{2*n^2} = \Ode{n^2}.
		
	\end{subsection}


	\begin{subsection}{Pruebas y Resultados}
		
		\gra{secuencia_unimodal/count_test_unimodal.pdf}
		Textooo

		\gra{secuencia_unimodal/count_test_random.pdf}

	\end{subsection}

\end{section}



	\newpage

	%\begin{section}{Problema 2}

	\textit{El gobierno planea construir una ciudad. Lo único que falta es asignarle el sentido de circulación a las calles (puede asignarse sólo un sentido a cada una). Toda calle esta conectada a dos esquinas, y se quieren poder determinar si se puede llegar de cualquier esquina a cualquier otra.}
		
	\begin{subsection}{Explicación}
	Para obtener el resultado para el problema, se toma una esquina y se intenta volver a la misma recorriendo otras esquinas, transitando las calles de la posible ciudad. Si no podemos volver a la esquina de la que partimos, podemos concluir que el resultado del problema será: {\em ``No existe forma de asignar el sentido de circulación a las calles para poder llegar de cualquier esquina a cualquier otra'' }. Si podemos volver a la esquina desde la que partimos, podemos afirmar que todas las esquinas que forman parte del circuito, son esquinas accesibles de cualquier otra esquina de ese mismo circuito. Resta ver entonces, que para el resto de las esquinas que no forman parte de ese circuito, se puede partir de alguna de las pertenecientes al circuito y volver, transitando por alguna o todas de las esquinas que no pertenecían al circuito. 
	
	Este proceso se debe realizar para todas las esquinas, agrandando en cada paso el conjunto de esquinas que forman parte del circuito (siempre y cuando no podamos afirmar que \textbf{no} existe solución previamente). Si esto ocurre, la solución al problema será: {\em ``Existe forma de asignar el sentido de circulación a las calles para poder llegar de cualquier esquina a cualquier otra''. }
		
	\end{subsection}

	\begin{subsection}{Detalles de la implementación}
		Modelamos este problema mediante grafos donde los vértices corresponden a esquinas y las aristas a las calles que conectan dos esquinas.	

		Generamos la matriz de adyacencia del grafo, de $n\times n$ donde $n$ es la cantidad de esquinas. Cada posición $(i,j)$ de la matriz contiene un $uno$ si la esquina $i$ está conectada con la esquina $j$ por medio de una calle, y $cero$ en caso contrario. Nos referiremos a la matriz como $conexiones$.


		Para la resolución del problema recorrimos el grafo con una modificación del algoritmo $Depth\; First\; Search$. Nuestro algoritmo, en primera instancia, busca el primer ciclo que pueda encontrar. Si no llega a la solución y recorre todo los nodos del grafo, puede afirmar que el resultado es falso, ya que el grafo \textbf{no} es fuertemente conexo. En caso contrario, marca todos los nodos que componen el circuito simple como parte de los nodos que ya se encuentran conectados y entra a un ciclo ($while$). Este último, se encarga de buscar alguna arista que lleve a algún nodo todavía no perteneciente a la solución parcial. Si la encuentra, aplica $dfs$ hasta que vuelva al ciclo, si no la encuentra y todavía quedan nodos por recorrer, no hay solución (devuelve $falso$). El $dfs$ (en caso de que exista una arista libre) busca volver al ciclo original, y si lo hace, agrega todos los nodos por los que pasó como parte del ciclo. Vuelve a empezar hasta que haya pasado por todos los nodos ($devuelve\; fuertemente\; conexo$) o algún nodo no puede volver al ciclo ($devuelve\; no$). Cabe destacar que por cada vez que el $dfs$ encuentre un camino que sale y vuelve del ciclo, el nuevo ciclo crece (se agregan todos los nodos del camino resultante).\VSP

		\begin{pseudo}
		\func{ciudad}{$conexiones$}
		\tab $ciclo[0...n] \leftarrow false$\\
		\tab $encontre\_ciclo \leftarrow dfs\_primer\_ciclo(conexiones,ciclo)$\\
		\tab \FOR $j$ \TO $n$\\
		\tab \tab $termine \leftarrow termine$ \AND $ciclo[j]$\\
		\tab \WHILE $!termine$ \AND $encontre\_ciclo$\\
		\tab \tab $nodo\_busqueda,\;nodo\_salida \leftarrow adyacente\_extreno(coneciones,ciclo)$\\
		\tab \tab $conexiones[nodo\_busqueda][nodo\_salida] \leftarrow 0$\\
		\tab \tab $encontre\_ciclo \leftarrow dfs\_ciclo(conexiones,nodo\_busqueda,ciclo)$\\
		\tab \tab $conexiones[nodo\_busqueda][nodo\_salida] \leftarrow 1$\\
		\tab \tab $termine \leftarrow true$\\
		\tab \tab \FOR $i$ \TO $n$\\
		\tab \tab \tab $termine \leftarrow termine$ \AND $ciclo[i]$\\
		\tab \RET $termine$\\
		\end{pseudo}
	\end{subsection}


	\begin{subsection}{Análisis de complejidad}
		
	\end{subsection}


	\begin{subsection}{Pruebas y Resultados}

	\end{subsection}

\end{section}



	\newpage

	%\begin{section}{Problema 3}
	\textit{Dada una lista de ingresos y otra de egresos que contienen los horarios de ingreso y egreso de cada uno de los programadores de una empresa respectivamente, determinar la mayor cantidad de programadores que estan simultáneamente dentro de la empresa.}

	\begin{subsection}{Explicación}
		Para cada instancia tenemos una lista que contiene para cada programador su horario de ingreso a la empresa y otra con su horario de egreso. Además, tenemos guardado en cada momento la cantidad máxima de programadores en simultáneo.

Dado que ambas listas se encuentran ordenadas, nuestro algoritmo las recorre decidiendo a cada momento si se produce un ingreso o un egreso, es decir, si el horario que sigue en la lista de ingresos es anterior a la de egresos implica que hubo un ingreso, en caso contrario un egreso.

Cuando una persona ingresa a la empresa se incrementa el contador de la cantidad de programadores en simultáneo en el horario actual. Así, cuando se produce un egreso se compara si la cantidad de programadores dentro de la empresa previo a dicho egreso es mayor a la máxima cantidad de programadores en simultáneo hasta el momento, de ser así, actualizamos el máximo.

Luego se descuenta el recientemente egresado del contador parcial de cantidad de programadores en simultaneo.

Este procedimiento se repite hasta haber visto todos los ingresos, lo que nos garantiza tener el máximo correspondiente, ya que a partir de ese momento sólo se producirían egresos. Este comportamiento se ve reflejado en el siguiente pseudocódigo:

		\pagebreak
		Sea $n$ la cantidad de programadores, $j$ el índice dentro de la lista de ingresos y $k$ el índice dentro de la lista de egresos.
		
		\vspace{0.5cm}
		\begin{pseudo}
				\func{programadores\_en\_simultaneo}{$ingresos, egresos$}
(1)				\tab $max,tmp,j,k \leftarrow 0$ \\
(2)				\tab \WHILE($j< n$) &   \\
(3)				\tab \tab \IF($ingresos[j]\leq egresos[k]$) \\
(4)				\tab \tab \tab $tmp \leftarrow tmp+1$ \\
(5)				\tab \tab \tab $j \leftarrow j+1$ \\
(6)				\tab \tab \ELSE \\
(7)				\tab \tab \tab \IF($tmp>max$) \\
(8)				\tab \tab \tab \tab $max \leftarrow tmp$ \\
(9)				\tab \tab \tab $tmp \leftarrow tmp - 1$ \\				(10)				\tab \tab \tab $k \leftarrow k+1$ \\
(11)				\tab \IF($tmp>max$) \\
(12)				\tab \tab $max \leftarrow tmp$ \\
(13)				\tab \RET $max$
		\end{pseudo}




	\end{subsection}

	\begin{subsection}{Detalles de la implementación}
	Guardamos los horarios de ingreso de todos los programadores (de la misma forma que estan en el archivo de entrada, es decir, en orden creciente) en un arreglo de $strings$ (los cuales representan un horario en formato ``HH:MM:SS'') de tamaño $n$, donde $n$ es la cantidad de programadores. Además guardamos otro arreglo del mismo tamaño con los horarios de egreso.
	
	A medida que vamos recorriendo los arreglos $ingresos$ y $egresos$ necesitamos decidir si el horario de ingreso del programador $j$ es anterior o posterior al horario de egreso del programador $i$, esto lo hacemos comparando los $strings$ por menor o igual (que el horario de ingreso del programador $j$ sea el mismo que el horario de egreso del programador $i$ significa que ambos estuvieron en simultáneo en la empresa jústamente en ese horario ya que se considera que un programador permanece dentro de la empresa desde su horario de ingreso hasta su horario de egreso, incluyendo ambos extremos). Si la comparación resulta verdadera significa que el programador $j$ ingresa a la empresa por lo que incrementamos el contador de programadores en simultáneo en ese horario. En caso contrario lo decrementamos ya que el programador $i$ egresa. Antes de decrementar dicho contador verificamos si la cantidad de programadores en simultáneo previo al egreso de $i$ es mayor a $max$ (máxima cantidad de programadores en simultáneo calculada hasta el momento) y de ser necesario actualizamos $max$.

	Una vez que terminamos de recorrer la lista de ingresos, actualizamos $max$ ya que desde el último egreso visto se pueden haber producido nuevos ingresos. Una vez hecho esto tenemos determinada la mayor cantidad de programadores que estan simultáneamente dentro de la empresa.
	\end{subsection}

	\begin{subsection}{Análisis de complejidad}
			Elegimos el modelo uniforme para analizar la complejidad de este algoritmo porque el tamaño de los elementos es acotado y por lo tanto todas las operaciones elementales son de costo constante.\Pa
			
			Como cada programador tiene un ingreso y un egreso, tanto la lista de ingresos como la lista de egresos tienen longitud $n$.

El algoritmo en todos los casos recorre completamente la lista de ingresos, por lo que el peor caso es cuando el último ingreso y el último egreso corresponden al mismo programador, ya que para registrar éste último ingreso, tambien tuvo que recorrer toda la lista de egresos. Por este motivo, podemos inferir que a lo sumo se realizan $2n -1$ iteraciones. En cada una de estas tenemos un costo constante de operaciones, que no modifican la complejidad en el análisis asintótico. La complejidad algorítmica en el modelo uniforme es \Ode{n}.\VSP

		\noindent\textbf{En función del tamaño de la entrada}\\

			Como los elementos de ambas listas son de tamaño acotado, el tamaño de la entrada es proporcional a la longitud de las listas (n). Entonces la complejidad del algoritmo es \Ode{t}, donde $t$ es el tamaño de la entrada. La complejidad del algoritmo es lineal.
	\end{subsection}

	\begin{subsection}{Pruebas y Resultados}
	Las entradas utilizadas para probar correctitud están en el archivo\\ \texttt{pruebas.in}. Fueron elegidas para probar casos bordes como son: un solo programador, primero ingresan todos y luego empiezan a egresar, un ingreso se produce exactamente al mismo horario que un egreso, cada ingreso seguido del egreso correspondiente y por último ingresos y egresos mezclados.

	Por un lado, decidimos contar la cantidad de operaciones realizadas por el algoritmo en el mejor y pero caso. Para esto generamos las instancias de la siguiente manera:
	\begin{itemize}
		\item En ambos casos, elegimos un número máximo de programadores($m$) ($m=1000$ porque nos pareció suficiente para constrastar estos casos) y creamos una instancia $\forall\;n,0\leq n < m$ siendo $n$ la cantidad de programadores para esa instancia.
		\item Para las instancias que corresponden al mejor caso ingresan todos los programadores y luego egresan.
		\item Para las instancias que corresponden al peor caso los programadores que entran salen antes de que haya un nuevo ingreso, es decir, nunca va a haber más de un programador en simultáneo.
	\end{itemize}

	El objetivo de esta prueba es comparar ambos casos. Uno de ellos es el mejor caso para el algoritmo y esta acotado inferiormente por $n$ y el otro, el peor caso para el algoritmo se encuentra acotado superiormente por $n$. A pesar de que el costo del algoritmo es siempre $n$, se espera ver diferencias significativas en la cantidad de operaciones realizadas en uno u otro caso ya que en el primero de ellos es mínima porque al producirse todos los ingresos el algoritmo termina sin recorrer la lista de egresos y en el segundo máxima ya que para poder registrar el último ingreso tuvo que recorrer todos los egresos.
	
	Cada instancia correspondiente al peor caso se representa con una cruz roja mientras que cada instancia que corresponde al mejor caso está representada con una cruz azul. También está graficada en color verde la función $13*n$, ya que es la cota superior teórica previamente calculada y en amarillo la función $6*n$ que es la cota inferior calculada ambas con una constante aproximada calculada empíricamente.\VSP
	
	\gra{programadores/count_peor_mejor.png}\VSP

	Observamos que se cumple lo esperado, es decir, corroboramos empíricamente que lo analizado previamente tiene coherencia, ya que la diferencia en la cantidad de operaciones que realiza el algoritmo para resolver un problema cuya caracteristica es la del mejor caso con $n$ programadores se hace cada vez más significativa respecto a las del peor caso para un mismo $n$ a medida que dicho $n$ crece.\\
	
	Para los gráficos que siguen utilizamos un generador, que dado un número máximo de programadores $n$, genera instancias aleatorias con a lo sumo $n$ programadores. Corrimos el generador con $n$ igual a $cien$ dado que no necesitamos instancias demasiado grandes para poder apreciar el comportamiento del algoritmo.

	En los siguientes gráficos esperamos ver el comportamiento general del algoritmo en cuanto a cantidad de operaciones como en tiempo transcurrido por instancia.
	
	Las instancias se representan con una cruz roja, además en el gráfico de cantidad de operaciones aparece en color verde la función $13*n$ mientras que en el de tiempos aparece graficada en el mismo color $8,7*10^(-8)+4*10^(-7)*n$ por los motivos ya mencionados.\VSP

	\gra{programadores/count_test.png}\VSP
	
	\gra{programadores/time_test.png}\VSP

	En el primer gráfico se ve la función que dado el número de programadores, muestra la cantidad de operaciones realizadas por el algoritmo mientras que en el segundo se observa el tiempo consumido para procesar cada instancia, ambos en función de la cantidad de programadores.
	En los dos gráficos se puede observar que la función se encuentra acotada por una recta, con lo cual se ve sin problemas que se trata de un algoritmo que trabaja en "tiempo" lineal.

	Lo que obsevamos en los gráficos tiene coherencia con la complejidad teórica calculada. 
	En general, vemos que el algoritmo se comporta como \Thetade{n} ya que como mínimo recorre sólo toda la lista de ingresos y como máximo recorre tanto la lista de ingresos como la de egresos. Es decir, el costo del algoritmo esta acotado inferior y superiormente por $c*n$, con c constante.

	\end{subsection}

\end{section}








	
	\newpage
	%	\begin{subsection}{Anexo}
		\begin{subsubsection}{Correctitud ejercicio 1}
		La siguiente tabla representa la correspondencia entre las variables de entrada ($b$ y $n$) en 
		cada iteración del algoritmo implementado:

		\vspace{0.5cm}
		\begin{center}
		\begin{tabular}{|l|c|c|c|c|c|}
			\hline
			iteración   & $1$ & $2$           & $3$             & ... & $k$ \\
			\hline
			$n$         & $n$ & $\frac{n}{2}$ & $\frac{n}{2^2}$ & ... & $\frac{n}{2^{k-1}}$ \\
			\hline
			$b$         & $b$ & $b^2$         & $b^4$           & ... & $b^{2^{k-1}}$ \\
			\hline
		\end{tabular}
		\end{center}

		\vspace{0.5cm}
		\noindent Sean \\
		\indent
		\begin{tabular}{lp{6cm}}
			$A_k = \frac{n}{2^{k-1}}$ & la sucesión con los valores de $n$ en la iteración $k$, y \\
			$Z_k = impar(A_k) * b^{2^{k-1}} + par( A_k )$ $^{[1]}$ & la sucesión que tiene los valores de $b$ corres\-pondientes a los $n$ impares y $1$ en los pares
		\end{tabular} \\
		\vspace{0.2cm}
		entonces el cálculo hecho por el algoritmo está dado por 
		$$\displaystyle\prod_{i=1}^k Z_i = b^n$$

		Luego de cada multiplicación toma módulo $n$ ya que
			$$b^k * b^{n-k}\; mod\;n = ((b^k\;mod\;n)*(b^{n-k}\;mod\;n))\;mod\;n\;\forall\;k\leq n ^{[2]}$$ 

		De esta manera, incluso si el cálculo de $b^n$ es un número tan grande que no entra en el 
		tamaño de la variable, se va a poder realizar sin problemas (suponiendo que $(n-1)^2$ entra 
		en una variable).


		\vspace{0.5cm}
		\noindent{\footnotesize [1] $par(x)=1-impar(x)$\\ $impar$ se define como:
		\begin{displaymath}
			impar(x)=\left\{
			\begin{array}{ll}
				1 & $si $x$ es impar$ \\
				0 & $sino$
			\end{array}\right.
		\end{displaymath}
		} \\
		\vspace{0.5cm}
		{\footnotesize [2] Por propiedades del módulo $x*y\; mod \; z = ((x\; mod\; z)*(y\; mod\; z))\;mod\; z$ } \\
		\end{subsubsection}
	\end{subsection}

	
	\newpage
	
	\begin{section}{Mediciones}
		\begin{itemize}
			\item Para contar la cantidad aproximada de operaciones definimos una variable inicializada en $cero$ la cual incrementamos luego de cada operación.
			\item Para medir tiempo tenemos una función que ejecuta el algoritmo por un mínimo de tiempo pasado como parametro, esto lo hacemos para obtener una mejor precisión. Una vez que se cumple el tiempo tenemos en una variable la cantidad de veces que se ejecutó el algoritmo, luego dividimos el tiempo medido por $c$. 
		\end{itemize}
	\end{section}
	\begin{section}{Compilación y ejecución de los programas}
	Para compilar los programas se puede usar el comando \texttt{make} (Requiere el compilador \texttt{g++}).
	Se pueden correr los programas de cada ejercicio ejecutando \texttt{./secuencia\_unimodal}, \texttt{./ciudad} y \texttt{./prision} respectivamente.
		
	Los programas leen la entrada de stdin y escriben la respuesta en stdout. 		Para leer la entrada de un archivo \texttt{Tp1EjX.in} y escribir la respuesta en un archivo \texttt{Tp1EjX.out} ses puede usar:\\ \texttt{./(ejecutable) < Tp1EjX.in > Tp1EjX.out}
		
	Para medir los tiempos de ejecución: \texttt{./(ejecutable) time}. Devuelve para cada instancia el tamaño seguido del tiempo trancurrido (en segundo) para procesar esa instancia.

	Para contar la cantidad de operaciones: \texttt{./(ejecutable) count}. Devuelve para cada instancia el tamaño seguido de la cantidad de operaciones de cada instancia.
	\end{section}
	
	\newpage
	\begin{comment}
	\begin{section}{Conclusiones}
		Pudimos implementar las soluciones a los problemas propuestos utilizando diversas técnicas algorítmicas. En todos los casos las mediciones de tiempo y cantidad de operaciones se correspondieron con las complejidades teóricas calculadas. Esto indica que los modelos elegidos fueron adecuados.

		En el ejercicio $uno$ nos encontramos con un problema donde debíamos operar valores enteros muy grandes, esto hizo que no pudiésemos considerar constante el costo de las operaciones elementales. Tuvimos así que analizar la complejidad en el modelo logaritmico. Al analizarla, tuvimos que diferenciarla en función de $n$ y del tamaño de entrada (lo que se hizo en todos los ejercicios), con la particularidad de que en función de $n$ resultó logarítmico, y en función del tamaño de la entrada, polinomial.
		
		Además, pudimos observar que en este caso la limitación del algoritmo no está dada por el tiempo de ejecución sino por el rango de valores admitidos ya que a partir de cierto $n$ hay que trabajar con números más grandes de lo que se puede almacenar en la variable. Dicho esto, nos parece un buen algoritmo porque como cualquier otro que opera con variables numéricas en algun punto hace $overflow$ para una entrada 'suficientemente grande', y este puede devolver un resultado para todas las instancias -donde esto no ocurre- con un costo temporal muy bajo.

		En el problema $dos$ nos encontramos con un algoritmo de complejidad factorial que a partir de cierto tamaño de entrada 'relativamente chico' se torna inútil, ya que la ejecución para dichas instancias es extremadamente lenta y en la práctica muchas veces no es aceptable. Esto significa que el problema no está 'bien' resuelto para tamaños de entrada no necesariamente grandes, notándolo en las pruebas: a partir de $16$ chicas en casos 'malos' el algoritmo requeria de varias horas para devolver una respuesta.
		
		El problema $tres$ nos pareció un problema particularmente sencillo desde el punto de la implementación. Pudimos resolverlo con un algoritmo simple que tiene costo lineal tanto en función del tamaño de la entrada como de la cantidad de programadores. Se considera entonces un problema 'bien' resuelto desde el punto de vista computacional.
	\end{section}
	\end{comment}

\end{document}
