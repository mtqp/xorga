\begin{section}{Problema 2}

	\textit{El gobierno planea construir una ciudad. Lo único que falta es asignarle el sentido de circulación a las calles (puede asignarse sólo un sentido a cada una). Toda calle esta conectada a dos esquinas, y se quieren poder determinar si se puede llegar de cualquier esquina a otra.}
		
	\begin{subsection}{Explicación}
	Para obtener el resultado para el problema, se toma una esquina y se intenta volver a la misma recorriendo otras esquinas, transitando las clases de las posible ciudad. Si no podemos volver a la esquina de la que partimos, podemos concluir que el resultado del problema será: $"No existe forma de asignar el sentido de circulación a las calles para poder llegar de cualquier esquina a otra"$. Si podemos volver a la esquina desde la que partimos, podemos afirmar que todas las esquinas que forman parte del camino, son esquinas accesibles de cualquier otra esquina de ese mismo circuito. Resta ver entonces, que para el resto de las esquinas que no forman parte de ese camino, se puede partir de alguna de las pertenecientes al camino y volver, transitando por alguna o todas de las esquinas que no pertenecían al circuito.\\ 
	
	Este proceso se debe realizar para todas las esquinas, agrandando en cada paso el conjunto de esquinas que forman parte de un circuito (siempre y cuando no podamos afirmar que \textbf{no} existe solución previamente). Si esto ocurre, la solución al problema será: $"Existe forma de asignar el sentido de circulación a las calles para poder llegar de cualquier esquina a otra"$
		
	\end{subsection}

	\begin{subsection}{Detalles de la implementación}
				Modelamos este problema mediante grafos donde los vértices corresponden esquina y las aritas a los calles que conectan dos esquinas.\\	

		Generamos la matriz de adyacencia del grafo, de $n\times n$ donde $n$ es la cantidad de esquinas donde cada posición $(i,j)$ de la matriz contiene un $uno$ si la esquina $i$ está conectada con la esquina $j$ por medio de una calle, y $cero$ en caso contrario. Nos referiremos a la matriz como $conexiones$.\\


		Para la resolución del problema recorrimos el grafo de forma ordenada por niveles. Para esto hicimos una modificación al algoritmo $Breadth\; First\; Search$. La modificación consiste en visitar los vértices adyacentes al 'actual' tales que todavía no fueron visitados y tienen acceso permitido, cuando esto último no ocurre se pone el vértice en 'espera' hasta que por otro camino se encuentre al vértice que habilite el acceso al mismo. Al momento de conseguir el acceso a un vértice que se encuentra en 'espera' se lo accede directamente (sin volver a pasar por los vértices que llevan a él) considerando posible ese camino hacia el vértice $n$. Cabe destacar que este algoritmo puede acceder a cada vértice sólo una vez.\\ 
	\end{subsection}


	\begin{subsection}{Análisis de complejidad}
		
	\end{subsection}


	\begin{subsection}{Pruebas y Resultados}

	\end{subsection}

\end{section}

