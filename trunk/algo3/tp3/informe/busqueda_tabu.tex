\begin{section}{Tabu-Search}
	\begin{subsection}{Explicación}
		Finalmente implementamos una metaheuristica, es decir, una heurística que guía otra heurística, en este caso la búsqueda local del ejercicio anterior.

		El \texttt{Tabu-Search} permite evitar que la heurística de búsqueda local se estanque en óptimos locales cuando en realidad fuera de la vecindad existía una solución óptima global (mejor que la local).
		Para lograrlo, permite al algoritmo perseguir una solución peor que la mejor obtenida mediante búsqueda local, por una cantidad máxima de iteraciones.
		Pasada esta cantidad, consideramos que el algoritmo ya buscó lo suficiente y la mejor solución encontrada hasta el momento debe ser la óptima.

		Para no revisar las vecindades que se revisaron anteriormente (que son muy cercanas a la vecindad actual),
		cada vez que decidimos movernos a otra vecindad porque tenemos un nuevo aspirante a óptimo (más allá de que sea peor que la mejor solución que encontramos hasta el momento, pero lo llamamos así por su similitud con el mismo en la búsqueda local)
		prohibimos revertir el cambio que hicimos para llegar del aspirante anterior al nuevo, o sea, prohibimos volver a agregar el vértice que sacamos.

		Esto lo implementamos mediante un arreglo de vértices (los índices representan los vértices), donde para cada uno guardamos la cantidad de iteraciones que falta para que deje de ser tabú (el algoritmo lleva la cuenta de las iteraciones).
		Luego, cuando revisamos las vecindades de un aspirante, evitamos aquellas donde la modificación implica agregar un vértice tabú.

		Los parámetros que indican cuanto tiempo una variable queda tabú y cuantas veces se puede iterar sin encontrar un nuevo óptimo antes de cortar el algoritmo, los buscamos empíricamente viendo los resultados obtenidos con distintos parámetros, y son bla y bla respectivamente.

		CAMBIAR LOS PARAMETROS!!!
	\end{subsection}
	\begin{subsection}{Detalles de la implementación}
		A continuación, se muestra el pseudocódigo del algoritmo de heurística de búsqueda tabú.\\

		Sea $n$ la cantidad de vértices del grafo.

			\begin{pseudo}
				\func{busqueda\_tabu}{lista\_solucion,tamanyo,matriz\_adyacencia}
				\tab $\IF tamanyo==1 \OR tamanyo==n$\\
				\tab \tab $\RET tamanyo$\\
				\tab $grados[n] \leftarrow ordenar\_grados(matriz\_adyacencia)$\\
				\tab $\WHILE mejore$\\
				\tab \tab $iter\_tabu \leftarrow max(tamanyo,3)$\\
				\tab \tab $\FOR c \TO tamanyo$\\
				\tab \tab \tab $inicializar\;estructuras$\\
				\tab \tab \tab $\WHILE puedo\;seguir$\\
				\tab \tab \tab \tab $sacar\_de\_clique(actual,ultimo)$\\
				\tab \tab \tab \tab $tabu[ultimo] \leftarrow iter\_tabu$\\
				\tab \tab \tab \tab $formar\_completo(actual,matriz\_adyacencia)$\\
				\tab \tab \tab \tab $\IF tam\_actual>tamanyo$\\
				\tab \tab \tab \tab \tab $agrandar\_clique(actual,lista\_tabu)$
				\tab \tab \tab \tab \tab $tamanyo \leftarrow tam\_actual$\\
				\tab \tab \tab \tab \tab $c \leftarrow tamanyo$\\
				\tab \tab \tab \tab \tab $lista\_solucion \leftarrow lista\_actual$\\
				\tab \tab \tab \tab $\ELSE$\\
				\tab \tab \tab \tab \tab $restar\_tabu(tabu,lista\_tabu)$\\
				\tab \RET $tamanyo$\\
			\end{pseudo}

		La primer cláusula \texttt{if} verifica si la solución de búsqueda local encontró la clique tanto completa como la de un elemento. En ambos casos no tiene sentido aplicar el tabú seach ya que, si encontró el completo, esta solución no podrá ser mejorada, al contener todos los vértices. Si sólo encontró un vértice, implica que el de grado mayor en el grafo es de grado cero, por lo tanto, todos sus vértices son de grado cero. 

		En la implementación mantenemos la solución actual en una lista de vértices. La lista inicia ordenada de mayor a menor según los grados, esto lo hacemos para empezar a sacar desde el de menor grado ya que consideramos que es el que tiene más probabilidades de estar condicionando la clique. Además, tenemos un arreglo de tipo bool de tamaño $n$ donde el índice representa a los vértices y esta seteado en $verdadero$ si y sólo si el vértice pertenece a la clique actual. Mantenemos ambas estructuras porque usamos la lista para determinar el orden en que se eliminan (rotarla tiene costo constante) y el arreglo para verificar la pertenencia de un vértice a la clique (ya que esta operación en esta estructura de datos tiene costo constante).

		El valor de verdad de la guarda del \texttt{while} $puedo\_seguir$ viene dado por la conjunción entre $tam\_actual \neq 1$, $\neg mejore$ e $iteracion<n$.
		Pedimos que el tamaño de la clique actual sea distinto de $uno$ ya que nos interesa movernos a soluciones vecinas. Si el tamaño es $uno$ en esa iteración saca el último vértice de la clique por lo que perdemos referencia a la misma moviéndose inmediatamente al primer vértice segun la numeración que no esté tabú.
		Por otro lado, el \texttt{while} itera mientras no logre mejorar para forzar la salida del ciclo cuando encuentre una clique de mayor tamaño que la actual y así empezar a sacar vértices desde la primer rotación (ya que también se fuerza la salida del \texttt{for}).
		La última condición es para asegurar la salida del ciclo, ya que podría no mejorar nunca y ciclar entre diferentes cliques. Además, esto determina la cantidad de iteraciones que le permitimos buscar sin lograr mejorar, es un parámetro que ajustamos de la siguiente manera AJUSTEMOSLO ALGUN DIAAAAAA!!!!!ukuyfuyjtjft5umj76fu7666m,6b,ib7,ki7k,i87fk7k,KUFMKUTJYTDMHYTMJUY6FMJUYMKY

		\begin{itemize}			
			\item \texttt{rotar: } Dado que fijar la pertenencia de un vértice a la clique condiciona el resultado final, el orden en que se eliminen los vértices puede hacer la diferencia entre un buen resultado y uno malo, a pesar de encontrar un buen criterio para hacerlo. Por este motivo, decidimos empezar eliminando de menor a mayor grado, y en cada iteración rotar la lista para sacar los vértices en otro orden.
			
			\item \texttt{sacar\_de\_clique: }Esta función saca de la lista el último elemento y setea en $falso$ la posición correspondiente en el arreglo (dejando tabú la operación inversa (agregarlo a la clique) tantas iteraciones como el tamaño de la clique con la que empieza a sacar), es decir, excluye el vértice de la solución actual. Además, decrementa el tamaño de la clique, variable $tam\_actual$.
			
			\item \texttt{formar\_completo: } Esta función se encarga de buscar entre los vértices que actualmente no pertenecen a la clique e intenta agregarlos (agrega todo vértice que forma un completo con los ya pertenecientes), con el objeto de conseguir una de tamaño mayor. Para saber si agregarlo determina una solución al problema debemos verificar que forme un completo con los vértices ya incluídos. Para esto recorremos todos los vértices del grafo y para cada uno que pertenezca a la solución parcial chequeamos que sea adyacente al que pretendemos agregar. Si esto ocurre podemos agregarlo y agrandar la clique. Elegimos el vértice a agregar de mayor a menor grado.

			\begin{itemize}
				\item \texttt{Observación: } Al agregar condicionamos la clique resultante al igual que pasa al sacar sin hacer rotaciones (depende del orden en que lo hagamos la calidad de la solución). Es decir, encontrar una mejor solución depende del orden en que agreguemos los vértices, podriamos también hacer rotaciones para agregar pero esto aumentaría en $n$ la complejidad. Buscando un equilibrio entre eficiencia y calidad de la solución, decidimos que hacer ambas rotaciones (agregar, sacar) tenia una complejidad mayor a la que pretendemos aceptar, no hacer ninguno implica perdernos de encontrar mejores soluciones y obtener así soluciones muy precarias. Entonces elegimos arbitrariamente hacer las rotaciones sólo para sacar.

				\item \texttt{Observacion: } Si el algoritmo vuelve a la solución inicial y todavía le restan iteraciones del \texttt{while} anidado, queremos evitar que repita exactamente el mismo procedimiento pasando nuevamente por soluciones ya visitadas
			\end{itemize}

			\item \texttt{agrandar\_clique: } Esta función itera los vértices que están en la lista tabú e intenta agregarlos a la clique actual. Esto es porque como ya conseguí una solución mejor lograr agregar algún vértice que esta tabú contribuye aún más a la solución.

			\item \texttt{restar\_tabu: } Para cada vértice tal que tiene tabú mayor a $cero$ decrementa la cantidad de iteraciones que va a permanecer tabú. Si al decrementarlo deja de ser tabú lo elimina de $lista\_tabu$.

			Tanto en $formar\_completo$ como en $agrandar\_clique$ los vértices que agregamos a la lista los ponemos al principio de la misma, es una forma de poner tabú al menos tantas iteraciones como vértices había en la clique previo a agregarlos la operación inversa, sacarlos.
		\end{itemize}
	\end{subsection}

	\begin{subsection}{Desventajas}
		La numeración de los nodos toma un papel crucial a la hora de obtener una solución, es así que en casos patológicos puede pasar que:
		\begin{itemize}
			\item Como el algoritmo se mueve constantemente de solución, podría existir el caso donde vuelve a la solución inicial (ejemplo: la búsqueda local nos da un clique de dos vértices que se encuentran en un ciclo, y este ciclo está unido a un completo de tres o más). Si los nodos del completo tienen mayor numeración a los elementos del ciclo, el algoritmo ciclaría tomando en cada iteración (mientras no se cumpla la restricción de iteraciones pasadas como parámetros) como solución los vértices incidentes a una arista de ese ciclo, sin ver el completo.
			\item Para evitar soluciones ya exploradas (en el caso que exista un ciclo), verificamos si vuelve a la solución inicial, de ser así, se sale del ciclo y se procede a la siguiente rotación.
		\end{itemize}
	
	\end{subsection}

	\begin{subsection}{Complejidad temporal}
			En un pricipio el algoritmo inicializa el arreglo de los $grados$ y lo ordena, lo que tiene un costo de $n^2$ ya que se utiliza el algoritmo de ordenamiento $QuickSort$. Luego, obtiene la solución inicial mediante la heurística de busqueda local que como ya vimos tiene un costo de $n^4$.

			La función $rotar$ y $sacar\_de\_clique$ tienen costo constante ya que constan sólo de indexaciones, asignaciones y operaciones elementales sobre listas.

			La función $formar\_completo$ itera por todos los vértices y para cada uno de estos verifica si es factible agregarlo a la clique, lo cual tiene un costo de $n^2$ ya que dicha verificación consiste en recorrer todos los vertices del grafo (SE PODRIA MEJORAR ITERANDO LA LISTA!!!!).y ver que el vértice que se pretende agregar es adyacente a cada uno de estos. Además, se concatena la lista de los vértices agregados con la lista de la clique actual lo que tiene un costo constante.

			La función $agrandar\_clique$ tiene costo $n^2$ porque recorre todos los vértices tabú verificando si los puede agregar, la cantidad de estos vértices se puede acotar por $n$ (todos estén tabú) a pesar que la cantidad de iteraciones que un vértice permanece tabú depende del tamaño de la clique de la que se lo elimina y si el grafo contiene una clique de tamaño $n$, es decir, es un grafo completo, la clique máxima va a estar  determinada por la heurística constructiva y búsqueda tabú verifica en un principio que la solución inicial no sea todo el grafo. Para esto (saber si puede agregarlo), recorre nuevamente todos los vértices del grafo, corroborando que sea adyacente a cada uno de los pertenecientes a la clique.

			 La cantidad de iteraciones del primer ciclo \texttt{while} se puede acotar por $n$ (seria un caso hipotetico en el que la mejora sea sólo de un vértice, es decir, se inicie con la clique trivial (tamaño $uno$) y en cada iteración de este ciclo se logre incrementar en uno el tamaño de la clique). En cada una de estas iteraciones hay un ciclo \texttt{for} que se ejecuta a lo sumo tantas veces como el tamaño de la clique, también puede ser acotado por $n$, dentro de este ciclo se resetean las estructuras (las listas y los arreglos) lo que tiene un costo lineal. Además, hay un ciclo \texttt{while} (dentro del for) que se ejecuta a los sumo $n$ veces y en cada una de las iteraciones hay una llamada a la función $sacar\_de\_clique$ con costo constante y $formar\_completo$ con costo $n^2$, en caso de lograr mejorar hay una llamada a la función $agrandar\_clique$ con costo $n^2$, en caso contrario, hay una llamada a la función $restar\_tabu$ de costo lineal (recorre todos los vértices decrementando la cantidad de iteraciones que permaneceran tabú). Entonces la complejidad del \texttt{while} anidado es:  \Ode{n*(\Ode{sacar\_de\_clique} + \Ode{formar\_completo} + \Ode{agrandar\_clique} + \Ode{restar\_tabu})} = \Ode{n*(1+n^2+n^2+n)} = \Ode{n^3}.

			 Finalmente, se deduce que la complejidad final del algoritmo es $n^5$ ya que acotamos tanto la cantidad de iteraciones del cliclo \texttt{while} como del \texttt{for} por $n$, entonces tenemos \Ode{n*n*\Ode{while\_anidado}}=\Ode{n^5}.
	\end{subsection}

\end{section}
