\section{Desarrollo}
El desarrollo de este trabajo se dividi\'o en 3 partes: una implementaci\'on preliminar en lenguaje C, 
la implementaci\'on en lenguaje ensamblador cuyo desarrollo se bas\'o en la versi\'on de C, y la versi\'on definitiva 
que contiene agregados y arreglos diversos. Esta separaci\'on de etapas fue importante, ya que nos permiti\'o 
apoyarnos sobre un c\'odigo completamente funcional en C,  al momento de escribir las funciones en ensamblador.
A continuaci\'on detallamos las etapas del trabajo.


\subsection{Implementaci\'on preliminar en lenguaje C}
Al comienzo del trabajo se realiz\'o una versi\'on del mismo en C que act\'ue adecuadamente de acuerdo a lo pedido. 
Esta versi\'on buscaba implementar todas las funciones requeridas en lenguaje de alto nivel para luego utilizarla 
como gu\'ia para escribir el c\'odigo en \keyword{assembler}. \\

Una vez escrito el c\'odigo en C, \'este se aprovech\'o para ir reemplazando funci\'on a funci\'on con c\'odigo ensamblador, 
porque se pod\'ia asumir que el c\'odigo de alto nivel era correcto. Esto nos permiti\'o probar que las funciones 
reemplazantes se comportaban adecuadamente ya que deb\'ian igualar el comportamiento de las funciones reemplazadas.\\

Finalmente, todo el c\'odigo en C fue cambiado por sus respectivas versiones en \keyword{assembler} a excepci\'on del 
\code{main.c} el cual contiene todas las llamadas y la interfaz con el usuario que ser\'a detallada m\'as adelante. 
El c\'odigo C reemplazado esta incluido en los archivos \code{filters.c} y \code{filters.h}.


\subsection{Implementaci\'on en lenguaje ensamblador}
Se decidi\'o separar la implementaci\'on \keyword{assembler} en varios archivos espec\'ificos: uno por cada operador a 
implementar y otro (\code{apply\_mask.asm}) que aplica una mascara dada a la imagen a procesar. La idea es 
abstraerse de esa funci\'on en particular que es com\'un a todos los operadores y programarla aparte, para facilitar el 
testeo del c\'odigo.\\

Primero se escribi\'o la funci\'on \code{apply\_mask}, luego el c\'odigo general que comparten los tres operadores a 
implementar y por \'ultimo se utiliz\'o para generar los tres archivos fuente finales de los operadores.


\subsection{Versi\'on definitiva}
Finalmente, una vez terminado y testeado todos los archivos fuente, se agreg\'o una interfaz con el usuario por 
consola. La interfaz permite abrir una imagen, aplicarle un algoritmo a elecci\'on y guardarla en un archivo 
separado o bien abrir la imagen a procesar en una ventana y aplicar los operadores en tiempo real con la 
posibilidad de guardar el resultado en un archivo separado. \\
\indent Se tuvo especial cuidado en el manejo de los par\'ametros del programa para evitar errores y los errores 
correspondientes a problemas con la carga o guardado de la imagen tambi\'en fueron considerados.

\pagebreak
