\section{Conclusiones}

Con el presente trabajo se muestra que existen formas efectivas, eficientes y simples para el problema de aplicaci\'on 
de filtros espec\'ificos en el marco del procesamiento de im\'agenes. Dichas t\'ecnicas se resumen en un algoritmo simple, 
mas all\'a de que la teor\'ia detr\'as sea m\'as compleja. \\

En cuanto a lo implementado en el trabajo, se infiere directamente una mejora de rendimiento reemplazando el c\'odigo 
escrito en C, por el nuestro escrito en lenguaje ensamblador. M\'as all\'a de cuestiones  y decisiones tomadas que 
afectaron la velocidad final del algoritmo, la diferencia de tiempos entre el lenguaje de alto nivel y el de bajo 
nivel es notable. \\

A\'un as\'i la existencia de un algoritmo m\'as r\'apido y eficaz es evidente como demostr\'o la comparaci\'on de tiempos de 
nuestra implementaci\'on con la funci\'on propia de la librer\'ia. Dicho algoritmo posiblemente est\'e implementado de tal 
manera que aprovecha de mejor forma los registros del procesador, ahorra ciclos con otras instrucciones, o quiz\'a, 
procesando datos en paralelo mediante \code{MMX} o \code{SSE}. Pero implementar un algoritmo de ese estilo resultar\'ia en un c\'odigo m\'as 
complicado y dif\'icil de seguir, perdiendo el sentido original del trabajo. \\

Para finalizar cabe destacar que lo escrito fue dise\~nado para cambiar o agregar f\'acilmente el tipo de operador a 
utilizar en el futuro, respetando a su vez el prototipo de funci\'on dado en el enunciado y que se logr\'o crear una 
herramienta de realce de bordes en im\'agenes que sea general, veloz y efectiva  que al mismo tiempo puede llegar a 
ser \'util en la pr\'actica.
\pagebreak
