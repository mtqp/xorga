\section{Desarrollo}

	El desarrollo del trabajo se realizó en forma gradual, separando en cada ejercicio propuesto: Segmentación inicial, Paginación, Interrupciones 
y Manejo de Tareas. También se apoyó fuertemente sobre los archivos dados por la cátedra, que definían la estructura general del sistema, las dos tareas 
ya implementadas, macros y un archivo adicional para habilitar la \code{Gate A20}, necesaria para el correcto funcionamiento del trabajo.

	El archivo principal, que contiene la lógica más relevante del sistema, es el \code{kernel.asm}. Éste importará y usará los siguientes archivos:

\begin{itemize}
\item Segmentación y Paginación
	\code{
	\begin{itemize}
		\item gdt.c
		\item gdt.h 
		\item kernel-traductor\_paging.asm
		\item pintor\_paging.asm
	\end{itemize}
	}
\item Interrupciones
	\code{
	\begin{itemize}
		\item idt.c
		\item idt.h
		\item isr.asm
		\item isr.h
	\end{itemize}
	}
\item Tareas
	\code{
	\begin{itemize}
		\item tss.c
		\item tss.h
		\item pintor.tsk
		\item traductor.tsk
	\end{itemize}
	}
\end{itemize}

	Se crearon segmentos de código y datos de 4GB de tamaño, y se mapeo la memoria de vídeo. Luego se creó toda la paginación como fue definida 
en el enunciado del trabajo práctico. El manejo de interrupciones se basa en mostrar un mensaje cuando ocurre una excepción, indicando de que excepción 
se trata, y detener el proceso del sistema. 

	Después son agregadas mas entradas en la tabla de descriptores globales (\code{GDT}) para manejar el switching y ejecución de las tareas dadas, y se 
definen las \code{TSS} cuyo propósito es guardar el estado de ejecución de cada tarea. 

\pagebreak
