\section{Introducción}

El objetivo del siguiente trabajo es aplicar los conocimientos acerca de programación de sistemas operativos dados en clase. 
Se escribirá un sistema operativo cuyo propósito sea controlar dos tareas que se ejecutarán a la par.
Para ello, el sistema deberá definir todas los segmentos necesarios, usar paginación adecuadamente, escribir en memoria de vídeo, ser 
capaz de controlar interrupciones y finalmente, ejecutar las dos tareas dadas, asignándole tiempo del procesador a cada una y realizar 
intercambios entre ellas para crear la sensación de que se ejecutan al mismo tiempo.

Las tareas a ejecutar son el pintor y el traductor. El pintor escribe un mensaje que el traductor leerá y escribirá en pantalla. Cada tarea 
tiene especificada su dirección de lectura y escritura, la dirección de su pila, a donde escribe su mensaje, etc. 

Se utilizará el emulador Bochs para correr y probar el funcionamiento del sistema programado. El emulador posee la capacidad de debuggear 
convenientemente el código escrito.

\pagebreak
