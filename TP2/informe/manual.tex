\section{Manual de usuario}

\subsection{Ayuda r\'apida}

\noindent\textbf{Uso:} \\
\texttt{./bordes [opciones] [archivo] }\\

\noindent\textbf{Opciones:} \\
\texttt{
\begin{tabular}{ll}
-r\#	&	Aplica el operador \#\\
-g		&	Modo gr\'afico \\
\end{tabular}
}\\

\noindent \textbf{Operadores posibles:}\\
\indent 1: Operador de Roberts \\
\indent 2: Operador de Prewitt \\
\indent 3: Operador de Sobel derivando por X \\
\indent 4: Operador de Sobel derivando por Y \\
\indent 5: Operador de Sobel derivando por X e Y \\

Si no se especifica un archivo de entrada, se usar\'a 'lena.bmp' 

\subsection{Descripci\'on}

El programa se puede invocar en modo gr\'afico (\code{-g}) o directo (\code{-r\#}) y opcionalmente una imagen. En modo directo, 
se leer\'a una imagen pasada como par\'ametro o la imagen por defecto (\code{lena.bmp}), se le aplicar\'a el filtro seleccionado 
y se guardar\'a con el nombre original mas un postfijo que indica que filtro fue aplicado y con la extensi\'on original. \\

En modo gr\'afico, se puede abrir una imagen y aplicar los filtros en tiempo real, con las teclas del \code{1} al \code{5}, 
restaurar la imagen en escala de grises con la tecla \code{0}, y guardar el resultado actual con la tecla \code{s}. Se puede 
salir de este modo con la tecla \code{ESCAPE}. \\

El orden de los par\'ametros no importa, puede pasarse primero la direcci\'on de la imagen seguida por el modo a 
usar o viceversa. \\

\noindent Tipos de im\'agenes soportados:
\texttt{
\begin{itemize}
\item Windows bitmaps - BMP, DIB 
\item JPEG files - JPEG, JPG, JPE 
\item Portable Network Graphics - PNG 
\item Portable image format - PBM, PGM, PPM 
\item Sun rasters - SR, RAS 
\item TIFF files - TIFF, TIF
\end{itemize}
}
\noindent (extra\'ido de la documentaci\'on de la librer\'ia \keyword{OpenCv})

\subsection{Instrucciones de compilaci\'on}
Dirigirse a la carpeta del c\'odigo (\code{src/}) y ejecutar el comando \code{make}.\pagebreak
